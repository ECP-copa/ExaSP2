% PROGRESS library
PROGRESS is a FORTRAN library that can be used for general purpose quantum chemistry calculations. It implements several parallel rapid $O(N)$ and graph-based recursive electronic structure solvers and is publicly available at \url{https://github.com/lanl/qmd-progress}. As described above and shown in Fig.~\ref{scheme}, PROGRESS relies entirely on BML for algebraic operations, so while quantum chemistry algorithms and calculations are outlined in PROGRESS the mathematical manipulations are all performed in BML. This library is currently used by the LATTE and plans to use it in the \href{https://github.com/dftbplus/dftbplus}{DFTB+} are ongoing. The code is hosted on github \cite{bml} and integrated with travis-ci and codecov.io for continuous integration and code coverage analysis. Every commit is tested over a set of compiler and compiler options.


The following is an example of how a high-level pseudocode using progress ligrary sould look like.

\begin{algorithm}[H]
  \algrenewcommand\algorithmicfunction{\textbf{program}}
  \begin{algorithmic}
    \parskip 0.05cm
    {\fontsize{0.3cm}{0.3em}\selectfont 
      \Function{HuckelDOS}{} \\
        \State \textbf{use} \verb|bml| \comm{!BML lib.} \\
        \State \comm{!PROGRESS lib modes.}        
        \State \textbf{use} \verb|prg_system_mod, prg_dos_mod, prg_density_mod, prg_huckel_mod| \\
        %
        \State \textbf{call getarg}(\verb|1,coordsfile|) \\
        %
        \State \textbf{call} \prog{prg\_parse\_system}(\verb|system,coordsfile,"pdb"|) \comm{!Parsing coords file.}\\
        %
        \State \textbf{call} \prog{prg\_get\_hshuckel}(\verb|ham_bml,over_bml,coordinates, ... |)\\
        \State \textbf{call} \prog{prg\_orthogonalize}(\verb|ham_bml,over_bml,orthoham_bml, ... |)\\
        \State \textbf{call} \prog{prg\_get\_eigenvalues}(\verb|orthoham_bml,eigenvals ... |)\\
        \State \textbf{call} \prog{prg\_get\_tdos}(\verb|eigenvals,"tdos.out" ... |)
        %
      \EndFunction
    }       
  \end{algorithmic}
  \caption{Pseudocode showing the use of calls to the PROGRESS library. In this example we compute the total density of state of a given molecular system using and extended H\"{u}ckel physical model.}
  \label{huck}    
\end{algorithm}


In algorithm \ref{huck} whe show how the progress fucntions are utilized to optain the denstiy of states (DOS) of a particular molecular system, starting from the coordinates. 
This high-level script gives back a file \verb!tdos.out! with the total density of states. It first parses the input file (\prog{prg\_parse\_system}) containing the coordinates which is a character contained in the variable (\verb!coordinates!). Then it constructs the Hamiltonian and overlap matrix using in this case an extended H\"{u}ckel physical model (physical models are not distributed with the progress library and this call is just addded as an example) (\prog{prg\_get\_hshuckel}). It orthogonalizes the Hamiltonian (\prog{prg\_orthogonalize}). It computes the eigenvalues of the system (\prog{prg\_get\_eigenvalues}) and finally it constructs the total DOS (\prog{prg\_get\_tdos}). Although simple, the latter is a very ilustrative example of how a code that uses the PROGRESS libreary looks like. 

