% QMD Program flow

The QMD capabilities are included in a computational framework aiming to foster developments in computational chemistry packages. The framework relies on two main libraries (PROGRESS and BML) were the computational developments are performed. In the BML library, the basic matrix operations are taken into account. Essentially, these operations consist of linear algebra matrix operations which are optimized based on the format of the matrix and the architecture where the program program will run. At the intermediate level, the PROGRESS library contains all the solvers (new techniques) that are developed where every matrix operation is taken care by calling the BML library. At the highest end of the calls we have the ``application code'' which is the code that will benefit from the new techniques developed at the PROGRESS level. 
% 
\vspace{0.5cm}
\begin{figure}[htpb]
   \begin{minipage}[t][2.0cm]{\textwidth}
    \begin{center}
      \smartdiagramset{
%         uniform color list=red!60!black for 1 items,
        set color list={red!50, green!50,blue!60},
        back arrow disabled=true,
        module minimum width=2cm,
        text width=2cm,
        arrow style=->,
        additions={
          additional item offset=0.70cm,
          additional item border color=black,
          additional connections disabled=false,
          additional arrow color=red,
%           additional arrow tip=stealth,
%           additional arrow line width=1pt,
          additional item bottom color=black!50,
        }
      }
        \smartdiagramadd[flow diagram:horizontal]{
            Application Code,PROGRESS,Basic Matrix Library
        }{}
    \vspace{0mm}\par
        
    \end{center}
   \end{minipage}    
 
   \begin{minipage}[t][2.0cm]{\textwidth}
    \begin{center}
      \smartdiagramset{
        set color list={black!50, blue!60},
        back arrow disabled=true,
        module minimum width=2cm,
        text width=2cm,
        arrow style=->,
        additions={
          additional item offset=0.70cm,
          additional item border color=black,
          additional connections disabled=false,
          additional arrow color=red,
          additional item bottom color=black!50,
        }
      }
        \smartdiagramadd[flow diagram:horizontal]{
            Proxy app,Basic Matrix Library
        }{}
    \vspace{0mm}\par
        
    \end{center}
   \end{minipage}    
   
   \vspace{-0.5cm}
      
  \caption{Top: Three-level abstraction model for developing in quantum chemistry. \textcolor{blue}{Basic Matrix library (BML)}: Only linear algebra or mathematically related operations are performed. The operations are made available through the BML application programming interface.  
  \textcolor{green}{PROGRESS library}: This is an archive of new capabilities offered as modules that rely entirely on BML. Including graph based parallel e-structure solvers and recursive Fermi operators expansion using fast linear scaling algebra.  
  \textcolor{red}{Application code}: The high-level code that computes a quantum based chemical property. The application code could also be any available code such as: \href{https://github.com/lanl/LATTE}{LATTE}, \href{http://yaehmop.sourceforge.net/}{YAEHMOP}, \href{http://www.dftb-plus.info/}{DFTB+}, \href{http://departments.icmab.es/leem/siesta/}{SIESTA}, etc.
  Bottom: Scheme showing an extracted Proxy app calling the bml library.
  }
  \label{scheme}
 \end{figure}
  
 A proxy application can be constructed from the algorithm in hand. By doing this we aim at optimizing the algorithm as much as possible given the computer architecture. Once this has been done, the optimized proxy app can be added to the PROGRESS library to be used by any application code (See figure \ref{scheme} bottom). 
