% DFTB+
The tight-binding (TB) method is a very popular method to determine the electronic structure of a 
physical-chemical system. It has been applied to a wide range of systems, ranging from insulators to
transition metals. Despite not been as accurate as the \textit{ab initio} methods, it provides
the possibility of treating systems with more atoms, or to study the behavior of phenomena
for longer periods of times when dynamical studies are carried out.
%
In quantum chemistry, the most accurate results are provided by \textit{ab initio} methods in which the only approximation is given by the use of a non-complete basis set and the choice of the correlation and exchange functionals in the case of density functional theory (DFT). 
However, one of the most distinct disadvantage in the application of these methods has been
the impossibility of dealing with extensive systems of the order of several thousands of atoms. This is
because the complexity of the calculation increases proportionally to the cube of the number of
atoms in the system to be treated.

There is no simple theory or model to describe the TB method, but they can all be seen as simplifications of 
DFT. Typically, atomic site energies $\bra{\phi_i}\hat{H}_{ii}\ket{\phi_i} = \epsilon_i$ and couplings $\bra{\phi_i}\hat{H}_{ij}\ket{\phi_j} = V_{ij}$ are available from experiments or calculations using high-level physical models. Sometimes even \textit{ad hoc} parameter are use to explore the behavior of a model system under different regimes \cite{Negre2008-2}. 
%
The TB parameters give a very tangible idea of the constitution of the chemical bond
and are used to construct the Hamiltonian matrix $H$.
The H\"{u}ckel method, for example, is a typical example of a tight-binding semi-empirical method.

In general, if each ``tight binding site'' is associated with a particular atom, a minimal basis set of Slater orbitals is used. The latter constituted at most by an ``s'' three ``p'' and five ``d'' orbitals for each atom. The valence wave functions constructed with
these orbitals are part of the bond in most materials. Transition metals are characterized by having s or d orbitals participating on the chemical bond. Many of the materials that are not constituted by transition elements (insulators or semiconductors)
are well represented by using only s and p orbitals.

The accuracy of the tight-binding methods depends on the degree of approximation of the model.
The need to making computations faster leads to approximations such as: Ignoring the integrals of two and three centers, assuming that the
atomic orbitals are orthogonal, etc.

One of the determinants of the quality of a TB model is what is known as
transferability. This means that the model in question can be transferred to other
systems without diminishing the original predictive power. The numerical complexity
of TB methods increases, usually with $N^3$ being $N$ the number of orbitals
atomic system. The order of the method, in this case, is the power with which it increases
the numerical complexity. For metal systems it is difficult to reduce the order of the method,
because the elements of the Hamiltonian matrix that are outside the diagonal, have significant values.

The extension of TB to diverse chemical systems was possible after the introduction of the density functional tight-binding (DFTB) method by Eltsner and coworkers in 1998 \cite{Elstner1998}. This tight-binding method is becoming more and more popular and widely used in chemistry, biochemistry, material science, etc. In DFTB theory the matrix elements of the Hamiltonian are not computed directly but they are optimized based either on experimental data or calculations with a higher level of theory. Since first introduction by Thomas Frauenheim \cite{Thomas} in 1998 \cite{Elstner1998}, the DFTB technique has been adopted by several groups and the number of publications where this technique is used are increased year by year. Moreover, the DFTB+ code, a sparse implementation of DFTB\cite{Houraine2007} is used by several reaserch groups and private industries around the world. Examples of applications of DFTB span from energy related material science to biochemistry applications \cite{Elstner2014}.
The DFTB formalism can be considered as an approximation to standard density functional theory (DFT). The Hamiltonian matrix elements are parametrized as a function of the interatomic distance, allowing for a fast construction of matrix $H$. In general, speedups can be two or three times when compared to \emph{ab-initio} methods such as DFT. \cite{Kosk2009}

One of the main bottlenecks of a quantum chemistry calculation is the solution of the generalized eigenvalue problem which scales with the cube of the system size $\mathcal{O}(N^3)$. The Theoretical and Computer sciences divisions at Los Alamos National Laboratory have accumulated almost 15 yeas of experience addressing this issue since the first method to compute the Density matrix (without having to solve the eigenvalue problem) was introduce by Niklasson in 2002 \cite{Niklasson}. This method is called the second order spectral projection (SP2) and consists of a ``purification'' of the eigenvalues based on an iterative expansion of the Fermi operator. The latter rendered the calculation of the Density matrix $\rho$ linear with the function of the system size $\mathcal{O}(N)$ caring over several advantages respect to traditional diagonalization of the Hamiltonian matrix. Another important breakthrough that followed the SP2 method is the discovery of the extended Lagrangian (XL) formulation\cite{Niklasson2006, Niklasson2008} that allows to reduce the number of self-consistency cycles up to only one density matrix calculation during quantum molecular dynamics (QMD) simulations or geometry optimization. 

These two techniques are essential to perform linear scaling QMD with a low prefactor and will open up lots of possibilities in many different areas of chemistry as they allow to simulate larger system sizes and longer time scales.
An implementation of the SP2+XL technique is available in the LATTE\cite{Sanville2010} code offering the possibility of performing several type of calculations. Initially, this code was developed to compute the effect of shock wave compression induced reaction\cite{Cawkwell}, but its use has been extended to several areas of material sciences for projects at LANL. The technique hereby mentioned allows for example, for the simulation of an entire solvated protein with more than 10,000 atoms \cite{Mniszewski2015}. 
The \texttt{PROGRESS}
library contains a recompilation of all the techniques implemented in LATTE with the advantage that these technique can now be ported to any open source available computational code and the three level computational framework allows to foster collaboration between different fields of applied computational science. 

As part of 2015 collaboration between LANL and Bremen Center for Computational Materials Science (BCCMS) a first implementation of the SP2 method 
in the DFTB+ code has been done. A new  LGPL\-icensed open source release of DFTB+ allows now for a full integration of the code using PROGRESS and BML.
DFTB+ will hence benefit of all the advances perform on this ECP project. 


% 
%  




